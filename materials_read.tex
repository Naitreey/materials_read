% arara: xelatex: {action: nonstopmode,synctex: yes}
% arara: biber
% arara: xelatex: {action: nonstopmode,synctex: yes}
% arara: xelatex: {action: nonstopmode,synctex: yes}
\RequirePackage[l2tabu, orthodox]{nag}
\documentclass{article}

%文献宏包
%\usepackage[style=numeric,backend=biber]{biblatex}

\usepackage[a4paper]{geometry}

%字体宏包
%\usepackage[Ligatures=TeX]{fontspec}
%中文宏包
%\usepackage[UTF8,winfonts,fancyhdr,hyperref,fntef]{ctex}
\usepackage{xeCJK}

\usepackage[pdfborder={0 0 0},colorlinks=false]{hyperref}
\usepackage{graphicx,amsmath,amssymb,bm,mathtools,subcaption,caption,cleveref}
\usepackage{IEEEtrantools}
\usepackage{dtklogos}

%数学定义
\renewcommand{\emptyset}{\varnothing}

%single & double quotes
\newcommand{\sq}[1]{`#1'}
\newcommand{\dq}[1]{``#1''}

%文献源 \addbibresource{•}

\newcommand{\book}[1]{\textit{#1}}

%设置字体
%\setmainfont[Ligatures=TeX]{Helvetica}
%\setsansfont{Helvetica}%Arial
%\setmonofont{}
\setCJKmainfont[BoldFont={SimHei},ItalicFont={楷体}]{SimSun}%Hiragino Sans GB W3/W6
%\setCJKsansfont{Hiragino Sans GB W3}
\setCJKmonofont{幼圆}

%代码宏包
%\usepackage{listings}
%\lstset{language=,frame=single,texcl=true,mathescape=true}%[LaTeX]TeX or mathematica or etc?

\title{Materials That I Have Read All These Years}
\author{Naitree Zhu}
\date{Last modified \today}

\begin{document}
\maketitle
%目录 \tableofcontents

\section{Physics}
\subsection{经典力学}
\begin{itemize}
    \item \book{普通物理学教程: 力学} (第二版), 漆安慎, 杜婵英. (在读)
    \item \book{新概念物理教程: 力学} (第二版), 赵凯华, 罗蔚茵. (在读)
    \item \book{An Introduction to Mechanics}, 2nd, by Kleppner and Kolenkow. (在读)
    \item \href{http://physics.stackexchange.com/}{Physics.SE}
        \begin{itemize}
            \item \href{http://physics.stackexchange.com/questions/1984/why-does-holding-something-up-cost-energy-while-no-work-is-being-done}{Why does holding something up cost energy while no work is being done?}
            \item \href{http://physics.stackexchange.com/questions/132978/who-does-work-while-walking}{Who does work while walking?}
        \end{itemize}
\end{itemize}

\subsection{热物理学}
\begin{itemize}
    \item \book{新概念物理教程: 热学} (第二版), 赵凯华, 罗蔚茵. (在读)
    \item \book{热物理学基础}, 包柯达. (在读)
    \item \book{热学}, 秦允豪. (在读)
    \item \book{Heat and Thermodynamics}, 7th edition, by Zemansky. (在读)
    \item \book{Thermodynamics and An Introduction To Thermostatistics}, 2nd edition, by Callen. (在读)
\end{itemize}

\subsection{电磁理论}
\begin{enumerate}
    \item 普通物理学教程: 电磁学, 第二版, by 梁灿彬. (在读)
\end{enumerate}
\subsection{量子理论}
\begin{itemize}
    \item \href{http://physics.stackexchange.com/}{Physics.SE}
        \begin{itemize}
            \item \href{http://physics.stackexchange.com/questions/133270/can-someone-please-qualitatively-explain-unitary-group-from-a-physics-perspectiv}{Can someone please qualitatively explain unitary group from a physics perspective?}
        \end{itemize}
\end{itemize}
\section{Mathematics}
\subsection{General Materials}
\begin{itemize}
    \item \href{http://mathworld.wolfram.com/}{Wolfram MathWorld}
    \item Wikipedia
    \item Mathematics Stackexchange
    \item Mathoverflow
\end{itemize}
\subsection{分析学}
\begin{itemize}
    \item \book{高等数学}, 同济第六版.
    \item \book{数学分析新讲}, 第一册, 张筑生. (在读)
\end{itemize}
\section{Computer Science}
%
\subsection{Learned Languages}
\begin{itemize}
    %programming languages
    \item C
    \item Python
    \item Java
    \item Javascript
    \item Mathematica
    \item shell script
    \item AWK
    \item TeX
    %markup languages
    \item TeX
    \item HTML
    \item CSS
    \item reStructuredText
    \item Markdown
    %database languages
    \item SQL
    \item MongoDB
    %data format
    \item XML
    \item YAML
    \item JSON
\end{itemize}
%
\subsection{Algorithm}
\begin{enumerate}
    \item Introduction to Algorithms, by Cormen. (在读, 很多有价值的参考资料在章末)
\end{enumerate}
\subsection{Design Patterns and Refactoring}
\begin{enumerate}
    \item Refactoring: Improving the Design of Existing Code (在读)
\end{enumerate}
\subsection{TeX}
\begin{itemize}
    \item \LaTeX{} Wikibook.
    \item \LaTeX{} Stackexchange
    \item The Not So Short Introduction to \LaTeXe{}.
    \item arara user manual. (无笔记)
    \item listings package documentation. (未读完)
    \item geometry package documentation.
    \item pdfpages package documentation. (未读完)
    \item amsmath package documentation.
    \item siunitx package documentation.
    \item lnotes, by 包太雷.
    \item lnotes2, by 包太雷.
    \item booktabs package documentation.
    \item \href{http://tex.stackexchange.com/}{TeX.SE}
        \begin{itemize}
            \item \href{http://tex.stackexchange.com/questions/198613/spacing-around-align-environments}{Spacing around align environments}
            \item the TeXbook. (在读)
            \item pgfmanual. (在读)
            \item enumitem package documentation. (在读)
            \item cool package doc \& demo. (在读)
        \end{itemize}
\end{itemize}
\subsection{Vim}
\begin{itemize}
    \item Vim tutor.
    \item Vim documentations
        \begin{itemize}
            \item usr_40.txt (在读)
            \item if_cscop.txt (cscope)
        \end{itemize}
    \item Vim-LaTeX reference. (剩下后面需要理解 vimscript 的部分)
    \item Vim-LaTeX reference card.
    \item \book{Practical Vim}, by Drew Neil. (在读)
    \item \href{http://vim.wikia.com/wiki/Vim_Tips_Wiki}{Vim Tips wiki}
    \item Vim questions on StackOverflow
    \item vim features and commands:
        \begin{itemize}
            \item movement
                h, j, k, l, w, W, e, E, \^, \$, gg, G, {n}G, /{char}, ?{char}, n, N, *,
                <C-o>, <C-i>, f, F, t, T, ;, \verb|,|, zz,
                \end{itemize}
            \item editing
                \begin{itemize}
                    \item cmdline mode
                        :w, :wa, :wq, :wqa, <C-p>, <C-n>, \%, :h
                    \item normal mode
                        a, A, I, i, x, d, dd, D, gU, gu, u, <C-r>, r, R, c, C, s, S,
                        v, V, <C-v>, o, O, y, yy, Y, >, >>, <, <<, ., <C-a>, <C-x>
                    \item insert mode
                        <C-k>, <C-v>, <ESC>,
                \end{itemize}
            \item info
                <C-g>, ga,
            \item session
                <C-z>
        \end{itemize}
\end{itemize}
\subsection{Mathematica}
\begin{itemize}
    \item \book{Mathematica 与大学物理计算}. (完全不合理地使用了 MMA)
    \item \book{The Mathematica Book}, by Stephen Wolfram. (在读)
    \item Mathematica virtual book---help doc. (在读)
    \item Mathematica Cookbook, by Sal Mangano (在读)
    \item Power programming with Mathematica: The kernel. (在读)
    \item Video tutorials: \href{http://www.wolfram.com/broadcast/}{Wolfram Screencast \&{} Video Gallery} (在读)
    \item \href{http://demonstrations.wolfram.com/}{Wolfram Demonstrations Project} (在读)
    \item \href{http://stoney.sb.org/wordpress/2008/12/adding-a-keyboard-shortcut-to-mathematica-v7/}{修改默认的 brackets 插入快捷键使得可以 wrap text selection}
    \item Mathematica Stackexchange:
        \begin{itemize}
            \item \href{http://mathematica.stackexchange.com/questions/18393/what-are-the-most-common-pitfalls-awaiting-new-users}{What are the most common pitfalls awaiting new users?}
            \item \href{http://mathematica.stackexchange.com/questions/24988/can-one-identify-the-design-patterns-of-mathematica}{Can one identify the design patterns of Mathematica?}
            \item \href{http://mathematica.stackexchange.com/questions/8310/using-the-mathematica-front-end-efficiently-for-editing-notebooks}{Using the Mathematica front-end efficiently for editing notebooks}
            \item \href{http://mathematica.stackexchange.com/questions/853/is-there-a-way-to-separate-variables-between-multiple-notebooks?lq=1}{Is there a way to separate variables between multiple notebooks?}
            \item \href{http://mathematica.stackexchange.com/questions/57656/how-draw-a-line-dynamically-using-manipulate}{How draw a line dynamically using manipulate?}
        \end{itemize}
    \item Wolfram blog:
        \begin{itemize}
            \item \href{http://blog.wolfram.com/2011/12/07/10-tips-for-writing-fast-mathematica-code/}{10 Tips for Writing Fast Mathematica Code}
        \end{itemize}
\end{itemize}
\subsection{C}
\begin{enumerate}
    \item C Primer Plus, 6th edition. (已读)
    \item Peter's gdb tutorial. (无笔记)
    \item gdb manual (在读)
    \item C11 Standard (在读)
    \item crt0 wiki \url{https://en.wikipedia.org/wiki/Crt0}
    \item C FAQs
        \begin{itemize}
            not to use \code{scanf} \url{http://c-faq.com/stdio/scanfprobs.html}
        \end{itemize}
    \item Why use apparently meaningless do-while and if-else statements in C/C++ macros?
        \url{http://stackoverflow.com/questions/154136/why-use-apparently-meaningless-do-while-and-if-else-statements-in-c-c-macros}
    \item Bit fields \url{http://en.cppreference.com/w/c/language/bit_field}
    \item libraries manuals
        \begin{enumerate}
            \item stdio.h(0P), stdio(3)
                \begin{itemize}
                    \item stdin, stdout, stderr
                    \item fopen(3), fdopen(3), freopen(3), fclose(3)
                    \item fileno(3)
                    \item feof(3), ferror(3), clearerr(3)
                    \item fread(3), fwrite(3)
                    \item fseek(3), ftell(3), rewind(3), fgetpos(3), fsetpos(3)
                    \item setvbuf(3), fflush(3)
                    \item printf(3), fprintf(3), sprintf(3), dprintf(3), snprintf(3)
                    \item vprintf(3), vfprintf(3)
                    \item scanf(3), fscanf(3), sscanf(3)
                    \item getchar(3), getc(3), fgetc(3), gets(3), fgets(3), ungetc(3)
                    \item putchar(3), putc(3), fputc(3), puts(3), fputs(3)
                \end{itemize}
            \item stdbool.h(0P)
            \item stdarg.h(0P), stdarg(3)
                \begin{itemize}
                    \item va_list
                    \item va_start, va_arg, va_end
                    \item va_copy
                \end{itemize}
            \item stdint.h(0P), inttypes.h(0P)
            \item float.h(0P)
            \item complex.h(0P), complex(7)
            \item stddef.h(0P)
            \item limits.h(0P)
            \item math.h(0P)
                \begin{itemize}
                    \item pow(3), isnan(3), isinf(3), fabs(3), sqrt(3)
                \end{itemize}
            \item tgmath.h(0P)
            \item ctype.h(0P)
            \item string.h(0P)
                \begin{itemize}
                    \item strlen(3)
                    \item strcmp(3), strncmp(3)
                    \item strcpy(3), strncpy(3)
                    \item strcat(3), strncat(3)
                    \item strchr(3), strrchr(3), strpbrk(3)
                    \item memcpy(3), memmove(3)
                \end{itemize}
            \item stdlib.h(0P)
                \begin{itemize}
                    \item malloc(3), calloc(3), realloc(3), aligned_alloc(3), free(3)
                    \item qsort(3)
                    \item atoi(3), atol(3), atoll(3)
                    \item strtol(3), strtoll(3)
                    \item strtof(3), strtod(3), strtold(3)
                    \item rand(3), srand(3)
                    \item mkdtemp(3), tmpfile(3)
                    \item exit(3), EXIT_SUCCESS, EXIT_FAILURE
                    \item atexit(3)
                    \item abort(3)
                \end{itemize}
            \item setjmp(3), setjmp.h(0P)
            \item errno(3), errno.h(0P)
            \item fenv(3), fenv.h(0P)
            \item time.h(0P)
                \begin{itemize}
                    \item time(2), RAND_MAX
                    \item clock(3), CLOCKS_PER_SEC
                \end{itemize}
            \item unistd.h(0P)
            \item wchar.h(0P), uchar.h
            \item wctype.h(0P), wctype(3)
            \item locale.h(0P)
            \item alloca.h
                \begin{itemize}
                    \item alloca(3)
                \end{itemize}
            \item assert.h(0P)
                \begin{itemize}
                    \item assert(3)
                \end{itemize}
            \item signal.h(0P), signal(2)
            \item termios.h(0P), termios(3)
            \item stdalign.h
            \item stdatomic.h, threads.h
            \item stdnoreturn.h
            \item iso646.h(0P)
            \item attributes(7)
            \item libpcap
            \item libguestfs
            \item libvirt
            \item Gnulib, GDSL, GLib, SGLIB
        \end{enumerate}
    \item source codes
        \begin{itemize}
            \item glibc source (在读)
            \item evince source (在读)
        \end{itemize}
    \item development tools
        \begin{itemize}
            \item cscope
                \begin{itemize}
                    \item Using Cscope on large projects \url{http://cscope.sourceforge.net/large_projects.html}
                    \item vim cscope doc: if_cscop.txt
                    \item cscope(1) manpage
                \end{itemize}
            \item jhbuild
        \end{itemize}
\end{enumerate}
\subsection{Java}
\begin{enumerate}
    \item Java: A Beginner's Guide (在读)
    \item Java Language Specification (在读)
    \item \href{http://docs.oracle.com/javase/8/docs/api/index.html?overview-summary.html}{Java Platform Standard Edition 8 API doc} (在读)
    \item \href{http://tomcat.apache.org/tomcat-8.0-doc/servletapi/}{Apache Tomcat Servlet API doc} (在读)
    \item \href{http://logging.apache.org/log4j/2.x/log4j-api/apidocs/index.html}{Apache Log4j API doc} (在读)
    \item How to Write Doc Comments for the Javadoc Tool (在读)
    \item Spring by Example (在读)
    \item Spring Framework Reference Documentation (在读)
    \item Tomcat tutorial (已读)
\end{enumerate}
\subsection{Python}
\begin{enumerate}
    \item tutorial --- Python official documentation (已读)
    \item Python Language Reference 2.7.10 --- Python official documentation (已读)
    \item Beginning Python From Novice to Professional (已读)
    \item PEP 0008 --- Style Guide for Python Code
    \item PEP 0273 --- Import Modules from Zip Archives
    \item PEP 0257 --- Docstring Conventions
    \item PEP 0302 --- New Import Hooks
    \item PEP 0318 --- Decorators For Functions and Methods
    \item PEP 0328 --- Imports: Multi-Line and Absolute/Relative
    \item PEP 0441 --- Improving Python ZIP Application Support
    \item PEP 3113 --- Removal of Tuple Parameter Unpacking
    \item PEP 3119 --- Introducing Abstract Base Classes
    \item Unifying types and classes in Python 2.2 \url{https://www.python.org/download/releases/2.2.3/descrintro/}
    \item Format String Syntax --- Python Library Reference
    \item Builtin exception hierarchy \url{https://docs.python.org/3.4/library/exceptions.html#exception-hierarchy}
    \item Unicode HowTo \url{https://docs.python.org/2/howto/unicode.html#python-2-x-s-unicode-support}
    \item Descriptor HowTo \url{https://docs.python.org/2/howto/descriptor.html}
    \item The Python Profilers \url{https://docs.python.org/2/library/profile.html}
    \item docutils
        \begin{itemize}
            \item docutils front-end tools \url{http://docutils.sourceforge.net/docs/user/tools.html}
        \end{itemize}
    \item modules/libraries/frameworks, and more:
        \begin{itemize}
            %multitasking
            \item thread (_thread in py3)
            \item threading (done: doc)
            \item multiprocessing (done: doc)

            \item pprint
            \item webbrowser
            \item random
            \item difflib
            \item csv

            %string
            \item string
            \item StringIO
            \item cStringIO

            %json
            \item json (doc, encode 时的 default 参数用法存疑, 它似乎只有在 JSON 标准数据类型之外作为输入时才起作用;)
            \item ujson

            %yaml
            \itme yaml

            %xml, html
            \item xml
            \item xml.dom.minidom
            \item xml.etree.ElementTree
            \item xml.etree.cElementTree
            \item html5lib
            \item pyquery
            \item xmltodict
            \item BeautifulSoup (bs4)
            \item Scrapy

            %cmdline
            \item colorama
            \item termcolor
            \item docopt (done: doc)
            \item getopt
            \item argparse
            \item cmd
            \item shlex
            \item glob
            \item fnmatch
            \item subprocess (done: doc)
            \item subprocess32
            \item readline
            \item rlcompleter

            %virtualization
            \item guestfs
            \item libvirt

            %system
            \item sys (done: doc)
            \item os (done: doc)
            \item shutil (done: doc)
            \item posixpath, a.k.a. os.path (done: doc)
            \item platform (done: doc)
            \item posix
            \item site
            \item signal
            \item psutil
            \item fileinput
            \item tempfile (done: doc)
            \item stat
            \item pwd
            \item getpass
            \item errno
            \item resource
            \item atexit

            %time
            \item time
            \item datetime

            %test
            \item trace
            \item coverage
            \item unittest
            \item doctest

            %refactor
            \item pylint

            %profiling
            \item timeit
            \item cProfile
            \item profile
            \item pstats
            \item pycallgraph

            %encryption
            \item hashlib
            \item base64
            \item binascii
            \item rsa

            %crypto
            \item Crypto
            \item ssl

            %compression
            \item zlib (done: doc)
            \item gzip
            \item tarfile
            \item zipfile
            \item zipapp (py3 only)

            %encoding
            \item codecs
            \item unicodedata
            \item chardet

            %ABC
            \item abc (done: doc, source)
            \item collections
            \item numbers (done: doc, source)

            % C and system-level interfaces
            \item array
            \item struct
            \item fcntl

            %data structure
            \item collections (done: doc, source (collections.py, _abcoll.py))
            \item heapq
            \item Queue

            %simple data store
            \item marshal
            \item shelve
            \item pickle (done: doc)
            \item cPickle (_pickle in py3)
            %database
            \item pymongo (mongodb)
            \item bson (mongodb)
            \item psycopg2
            \item sqlite3
            \item sqlalchemy

            %GUI programming
            \item wx (wxPython)
            \item PyQt

            %web programming
            %lower-level structure
            \item socket
            \item netifaces
            \item dpkt
            \item ipaddress
            \item pyroute2
            %client
            \item paramiko
            \item requests
            \item urlparse (urllib.parse in py3)
            \item urllib (urllib in py3)
            \item urllib2 (urllib in py3)
            \item pycurl
            \item httplib (http.client in py3)
            \item xmlrpclib
            %server
            \item wsgiref
            \item SocketServer
            \item BaseHTTPServer (http.server in py3)
            \item SimpleXMLRPCServer
            \item asyncore
            \item asynchat
            \item select (done: doc)
            \item cgi
            \item cgitb
            \item bottle (doc & source)
            \item uWSGI
            \item Twisted
            \item Django
            \item tornado
            \item shadowsocks
            %cookies
            \item Cookie (http.cookies in py3) (doc & source)

            %mimetype
            \item mimetype

            %template
            \item jinja2

            %email
            \item email
            \item smtplib
            \item smtpd
            \item poplib
            \item imaplib

            %SMB
            \item pysmb

            %snmp
            \item pysnmp

            %asn1
            \item pyasn1

            %scientific computing
            \item math
            \item SciPy
            \item NumPy
            \item IPython
            \item Sympy
            \item Matplotlib
            \item pandas

            %extension
            \item ctypes
            \item SWIG

            %package distribution
            \item distutils
            \item setuptools
            \item py2exe

            %documentation
            \item docutils
            \item Sphinx
            \item Read the Docs
            \item pydoc

            % image processing
            \item

            %import
            \item importlib
            \item imp
            \item zipimport
            \item pkgutil

            %checking
            %py2py3
            \item six
            \item future
            %codingstyle
            \item flake8

            %misc
            \item re (done: doc)
            \item yara
            \item pypdf2
            \item gc
            \item functools (doc)
            \item itertools
            \item operator
            \item keyword
            \item logging
            \item ConfigParser
            \item jsbeautifier
            \item textwrap
            \item uuid
            \item __future__ (done: doc, source)
            \item inspect
            \item types (done: doc, source)
            \item beanstalkc (done: official tutorial)
            \item pynsq
            \item traceback (done: 部分 doc and source code)
            \item copy
            \item py_compile
            \item warnings
            \item linecache
            \item dis
            \item weakref
        \end{itemize}
        \begin{enumerate}
            \item nginx tutorial \url{http://nginx.org/en/docs/beginners_guide.html}
            \item configuring nginx plus as a web server \url{https://www.nginx.com/resources/admin-guide/nginx-web-server/}
        \end{enumerate}
\end{enumerate}
%
\subsection{Go}
%
\subsection{Linux \&\ Unix}
\subsubsection{System Overview}
\begin{enumerate}
    \item Learning the Unix Operating System (无笔记)
    \item Unix in a nutshell (在读)
    \item Fedora 22 System Administrator's Guide (在读)
    \item RHEL6 deployment guide (在读)
    \item IPC wiki \url{https://en.wikipedia.org/wiki/Inter-process_communication}
\end{enumerate}
%
\subsubsection{User System}
%
\begin{itemize}
    \item sudo
        \begin{enumerate}
            \item \url{https://help.ubuntu.com/community/RootSudo}
            \item \url{https://help.ubuntu.com/community/Sudoers}
        \end{enumerate}
\end{itemize}
%
\subsubsection{System Structure}
%
\begin{itemize}
    \item How Linux Works, 2nd Edition (在读)
    \item Linux device mapper wiki \url{https://en.wikipedia.org/wiki/Device_mapper}
    \item Linux Logical Volume Manager (LVM) wiki \url{https://en.wikipedia.org/wiki/Logical_Volume_Manager_(Linux)}
    \item Power management/Suspend and hibernate wiki \url{https://wiki.archlinux.org/index.php/Power_management/Suspend_and_hibernate}
    \item upstart cookbook (obsolete, 作废)
    \item systemd
        \begin{enumerate}
            \item \url{http://0pointer.de/blog/projects/systemd-docs.html}
            \item systemd for Administrators:
                \begin{enumerate}
                    \item \url{http://0pointer.de/blog/projects/systemd-for-admins-1.html}
                    \item \url{http://0pointer.de/blog/projects/systemd-for-admins-2.html}
                    \item \url{http://0pointer.de/blog/projects/systemd-for-admins-3.html}
                    \item \url{http://0pointer.de/blog/projects/systemd-for-admins-4.html}
                    \item \url{http://0pointer.de/blog/projects/three-levels-of-off}
                    \item \url{http://0pointer.de/blog/projects/blame-game.html}
                    \item \url{http://0pointer.de/blog/projects/the-new-configuration-files}
                    \item \url{http://www.freedesktop.org/wiki/Software/systemd/}
                \end{enumerate}
            \item Predictable Network Interface Names \url{https://www.freedesktop.org/wiki/Software/systemd/PredictableNetworkInterfaceNames/}
        \end{enumerate}
    \item udev
        \begin{itemize}
            \item udev wiki \url{https://en.wikipedia.org/wiki/Udev}
            \item udev documentation \url{https://www.kernel.org/pub/linux/utils/kernel/hotplug/udev/udev.html}
            \item Writing udev rules \url{http://www.reactivated.net/writing_udev_rules.html}
            \item archlinux udev wiki \url{https://wiki.archlinux.org/index.php/udev}
            \item /usr/lib/udev/rules.d 规则 (在读)
        \end{itemize}
    \item D-Bus
        \begin{enumerate}
            \item D-Bus wiki \url{https://en.wikipedia.org/wiki/DBus}
        \end{enumerate}
    \item syslog
        \begin{enumerate}
            \item logger(1), syslog(3), rsyslogd(8), journalctl(1) manpages
            \item rsyslog documentation \url{http://www.rsyslog.com/doc/master/index.html}
            \item syslog wiki \url{https://en.wikipedia.org/wiki/Syslog}
        \end{enumerate}
    \item periodical jobs
        \begin{itemize}
            \item anacron wiki \url{https://en.wikipedia.org/wiki/Anacron}
        \end{itemize}
    \item iptables
        \begin{enumerate}
        \end{enumerate}
    \item SELinux
        \begin{enumerate}
            \item CentOS SELinux HowTo \url{https://wiki.centos.org/HowTos/SELinux}
        \end{enumerate}
    \item Pluggable Authentication Modules (PAM)
        \begin{enumerate}
            \item wiki \url{https://en.wikipedia.org/wiki/Linux_PAM}
            \item pam(8) manpage
            \item Understanding PAM Authentication and Security \url{http://aplawrence.com/Basics/understandingpam.html}
            \item pam.d(5) manpage
            \item pam_ftp(8) manpage
        \end{enumerate}
    \item PXE
        \begin{itemize}
            \item Preboot Execution Environment wiki \url{https://en.wikipedia.org/wiki/Preboot_Execution_Environment}
        \end{itemize}
    \item initramfs, dracut
        \begin{itemize}
            \item dracut wiki \url{https://en.wikipedia.org/wiki/Dracut_(software)}
            \item dracut kernel wiki \url{https://dracut.wiki.kernel.org/index.php/Main_Page}
            \item dracut kernel doc \url{https://www.kernel.org/pub/linux/utils/boot/dracut/dracut.html}
            \item wwoods' notes on dracut: theory, operation, and good practice \url{https://wwoods.fedorapeople.org/doc/dracut-notes.html}
            \item dracut source repo
        \end{itemize}
    \item sysfs
        \begin{itemize}
            \item The sysfs system by Patrick Mochel (在读, 涉及 kernel programming 的部分未读)
            \item kernel sysfs documentation https://www.kernel.org/doc/Documentation/filesystems/sysfs.txt
        \end{itemize}
    \item package management
        \begin{itemize}
            \item DNF system upgrade \url{https://fedoraproject.org/wiki/DNF_system_upgrade}
        \end{itemize}
\end{itemize}
%
\subsubsection{System Programming}
%
\begin{itemize}
    \item The Linux Programming Interface (在读)
    \item resource limit: \url{http://serverfault.com/questions/356962/where-are-the-default-ulimit-values-set-linux-centos}
    \item Upgrading: nuance about replacing executable \url{http://unix.stackexchange.com/questions/138214/how-is-it-possible-to-do-a-live-update-while-a-program-is-running}
    \item system calls
        \begin{itemize}
            \item open(), read(), write(), close(), lseek()
        \end{itemize}
    \item The Linux Kernel Archives --- Active kernel releases \url{https://www.kernel.org/category/releases.html}
\end{itemize}
%
\subsubsection{Shell Programming}
%
\begin{enumerate}
    \item Bash Reference Manual (已读)
    \item /dev/(tcp|udp)/ip/port as a weapon \url{https://securityreliks.wordpress.com/2010/08/20/devtcp-as-a-weapon/}
    \item Advanced Bash-Scripting Guide, by Mendel Cooper (在读)
    \item Learning the Bash Shell (在读)
    \item shell utils and commands:
        \begin{enumerate}
            \item bash, :, ., source, printf, echo, cd, pwd, type, umask, exit, if, case, select, read, trap, exec, eval, dirs, pushd, popd, ulimit, hash, [[ ]], test (\verb|[|), builtin, command, enable, shopt, set, unset, shift, time, times, alias, unalias, until, for, while, break, continue, coproc, function, return, declare, local, readonly, export, getopts, bind, complete, compgen, compopt, fc, history, !n, !-n, !!, !string, !?string[?], :0, :n, :^, :\$, :x-y, :-y, :*, :x*, :h, :t, :r, :e, :p, :s/old/new/, :gs/old/new/, jobs, fg, bg, kill, wait, disown, suspend, \%n, \%\%, \%string, \%?string, suspend character, caller, help, let, logout, mapfile, readarray
            \item sh, sudo, su, chsh, visudo, passwd, getty, login, chvt
            \item gnu readline
                move: backward-char (C-b) (<Left>), forward-char (C-f) (<Right>), backward-word (M-b), forward-word (M-f), beginning-of-line (C-a), end-of-line (C-e), vi-fWord, vi-bWord (M-B), character-search (C-]), character-search-backward (M-C-])
                delete: C-h (<Backspace>), delete-char (C-d) (<Del>), backward-kill-word (M-<Backspace>), kill-word (M-d), kill-line (C-k), unix-line-discard (C-u), unix-word-rubout (C-w)
                undo: undo (C-_, C-x C-u)
                clear screen: clear-screen (C-l)
                paste: C-y, M-y
                accept line: accept-line (C-j) (C-m) (<Enter>)
                search history: previous-history (C-p), next-history (C-n), reverse-search-history (C-r), forward-search-history (C-s), abort (C-g)
                eof: end-of-file (C-d)
                insert: quoted-insert (C-v), self-insert, insert-comment (M-#), shell-expand-line (M-C-e), edit-and-execute-command (C-x C-e)
                swap case: upcase-word (M-u), downcase-word (M-l), capitalize-word (M-c)
                completion: complete (Tab), menu-complete, complete-filename (M-/), complete-username (M-~), complete-variable (M-\$), complete-hostname (M-@), complete-command (M-!), complete-into-braces (M-\{)
                options: colored-stats, mark-directories, menu-complete-display-prefix,
                         show-all-if-ambiguous, skip-completed-text
                misc: re-read-init-file (C-x C-r), prefix-meta (ESC)
                cancel: ctrl-[ (ESC)
            \item mknod
            \item xinput, xclip,
            \item alternatives
            \item apt-get (install|update), dnf (install|remove|erease|update|updateinfo), yum, rpm
            \item ls, stat, cat, rm, mkdir, rmdir, mv, cp, less, vi, chmod, chroot, chown, touch, file, locate, head, tail, env, printenv, tr, groups, which, ln, readlink, uname, dd, dmesg, lsof, dirname, mktemp, printf
            \item id (man and info '(coreutils) id invocation')
            \item w, who, whoami, uptime, tty, whois (jwhois),
            \item seq
            \item pgrep, pkill, kill, kill, killall, ps, top, pidof, nice
            \item grep, bzgrep, xzgrep, zgrep, zipgrep, locale, find, xargs
            \item date, hwclock (clock)
            \item man, apropos, mandb, info
            \item texdoc
            \item shutdown, poweroff, reboot
            \item column, uniq, sort
            \item dmidecode
            \item sed, awk, gawk, cut, expect
            \item tar, gzip, gunzip, zcat, bzip2, bunzip2, bzcat, xz, unxz, xzcat, 7z, 7za, zip, unzip,
            \item md5sum
            \item dos2unix, unix2dos
            \item useradd, userdel, usermod, lid
            \item systemctl (status|start|stop|restart|enable|disable|list-units|list-unit-files|list-jobs|poweroff|reboot|suspend), systemd-analyze (blame|plot|dot), systemd-cgls, systemd-udevd, journalctl, logind.conf(5), systemd-journald.service(8), systemd-journald.socket(8), systemd-journald-dev-log.socket(8), /usr/lib/systemd/systemd-journald(8),
            \item service, run-parts, telinit, lsscsi, lsusb
            \item udev.conf(5), udevadm(8), udev(7), systemd-udevd.service(8)
            \item dracut(8), lsinitrd(1), mkinitrd(8), dracut.cmdline(7), dracut.conf(5), dracut.modules(7), dracut.bootup(7), dracut-cmdline.service(8),
            \item screen
            \item df, du, fdisk, gdisk, parted, gparted, mkfs.<type>, fsck.<type>, dumpe2fs, tune2fs, debugfs, mount, umount, samba, findmnt, blkid, lsblk, smartctl, smartd.conf, /etc/fstab (fstab(5)), /etc/mtab (mount(8)), /proc/mounts (proc(5)), sync, free, mkswap, swapon, swapoff, dmsetup, lspci, ionice, iotop
            \item console_codes(4)
            \item proc(5)
            \item grub2-install, grub2-mkconfig
            \item ssh, ssh-keygen, ssh-copy-id, sshpass sftp, ftp, scp, telnet, netcat (nc), wget, curl, rsync, nslookup, tcpdump
            \item hostname
            \item gcc, gdb
            \item ctags, cscope(1)
            \item make, diff, patch, ldd, strings, pmap, taskset, getopt
            \item gvim, vim, gvimdiff
            \item python, python3, pip, pip3, pydoc
            \item node, npm
            \item java, javac
            \item ping, iptables
            \item arp, arping
            \item ip (route|maddress|neighbour)
            \item traceroute(1)
            \item whois (jwhois)
            \item virsh (list|
                       create|start|shutdown|destroy|
                       dompmsuspend|dompmwakeup|
                       define|
                       capabilities)
            \item qemu-img (create|convert|info|snapshot)
            \item bluetoothctl
            \item vncviewer
            \item okular
            \item git (init|clone|branch|mv|status|remote|ls-remote|merge|mergetool|merge-base|merge-file|pull|fetch|push|commit|commit-tree|log|shortlog|checkout|show-branch|ls-files|ls-tree|read-tree|write-tree|hash-object|cat-file|rm|add|stash|config|var|diff|difftool|diff-tree|diff-index|apply|rebase|reset|revert|tag|show|instaweb|clean|cherry-pick|cherry|reflog|submodule|filter-branch|request-pull|format-patch|am|send-email|rev-parse|rev-list|rerere|describe|grep|blame|bisect|update-index|update-ref|symbolic-ref|archive|bundle|gc|prune|fsck|count-objects|help|credential|credential-cache|credential-cache--daemon|credential-gnome-keyring|credential-store|replace|update-server-info|send-pack|receive-pack|version), git-shell, gitignore(5), gitrevisions(7), gitattributes(5), githooks(5), gitcredentials(7), gitmodules(5)
            \item git annex (init|add|copy|move|describe|drop|dropunused|initremote|numcopies|unused|)
            \item svn (commit|checkout|log)
            \item mail
            \item sqlite3
            \item psql
            \item mongo, mongod, mongodump, mongorestore, mongoexport, mongoimport
            \item beanstalkd
            \item runc (start|spec|kill|list)
            \item docker (build|run|images|create|commit|rename|rm|rmi|ps|start|stop|kill|attach|export|logs|port|history)
            \item expressvpn,
            \item wdctl
        \end{enumerate}
    \item bash init procedures:
        \begin{itemize}
            \item /etc/profile
        \end{itemize}
    \item dd wiki \url{https://en.wikipedia.org/wiki/Dd_(Unix)}
    \item benchmark disk with dd \url{https://romanrm.net/dd-benchmark}
    \item docopt: Command-line interface description language \url{http://docopt.org/}
    \item customize terminal prompt (无笔记)
    \item description about p, x, etc. manpage sections: \url{http://unix.stackexchange.com/questions/204501/what-are-the-n-l-3pm-sections-of-the-manual-for}
    \item background process, daemon, etc
        \begin{itemize}
            \item background process on shell exit: \url{http://stackoverflow.com/questions/32780706/does-linux-kill-background-processes-if-we-close-the-terminal-from-which-it-has}, \url{http://superuser.com/questions/662431/what-exactly-determines-if-a-backgrounded-job-is-killed-when-the-shell-is-exited}, \url{http://unix.stackexchange.com/questions/3886/difference-between-nohup-disown-and#}, \url{http://unix.stackexchange.com/questions/4004/how-can-i-close-a-terminal-without-killing-the-command-running-in-it}
        \end{itemize}
    \item suid on interpreted programs: \url{http://unix.stackexchange.com/questions/364/allow-setuid-on-shell-scripts}
    \item AWK programming
        \begin{enumerate}
            \item The AWK Programming Language (在读)
        \end{enumerate}
\end{enumerate}
%
\subsubsection{Networking}
%
\begin{itemize}
\end{itemize}
%
\subsubsection{Desktop environment}
%
\begin{itemize}
    \item GNOME
        \begin{itemize}
            \item gnome wiki \url{https://en.wikipedia.org/wiki/GNOME}
            \item gnome newcomers guide \url{https://wiki.gnome.org/Newcomers/}
            \item gnome project tour \url{https://wiki.gnome.org/Newcomers/ProjectTour}
            \item gnome IRC \url{https://wiki.gnome.org/Community/GettingInTouch/IRC}
            \item Tools and tricks for solving tasks in a GNOME project
                  \url{https://wiki.gnome.org/Newcomers/FindAndSolveTasks}
            \item evince
                \begin{itemize}
                    \item evince wiki \url{https://en.wikipedia.org/wiki/Evince}
                    \item poppler wiki \url{https://en.wikipedia.org/wiki/Poppler_(software)}
                \end{itemize}
            \item jhbuild
                \begin{itemize}
                    \item build gnome: Set up JHBuild \url{https://wiki.gnome.org/Newcomers/BuildGnome}
                \end{itemize}
            \item GtkInspector \url{https://wiki.gnome.org/Projects/GTK\%2B/Inspector}
        \end{itemize}
\end{itemize}
%
\subsubsection{Virtualization}
%
\begin{itemize}
    \item general introduction
        \begin{itemize}
            \item Hardware virtualization wiki \url{https://en.wikipedia.org/wiki/Hardware_virtualization}
            \item Virtual Linux: An overview of virtualization methods, architectures, and implementations \url{https://web.archive.org/web/20080327111126/http://www-128.ibm.com/developerworks/linux/library/l-linuxvirt/?ca=dgr-lnxw01Virtual-Linux}
            \item Fedora Virtualization intro \url{https://fedoraproject.org/wiki/Virtualization?rd=Tools/Virtualization}
            \item Fedora Getting started with virtualization \url{https://fedoraproject.org/wiki/Getting_started_with_virtualization}
            \item hardware emulation wiki \url{https://en.wikipedia.org/wiki/Emulator},
                  full virtualization wiki \url{https://en.wikipedia.org/wiki/Full_virtualization},
                  hardware-assisted virtualization wiki \url{https://en.wikipedia.org/wiki/Hardware-assisted_virtualization},
                  paravirtualization wiki \url{https://en.wikipedia.org/wiki/Full_virtualization},
                  operating-system-level virtualization \url{https://en.wikipedia.org/wiki/Operating-system-level_virtualization}
            \item hypervisor wiki \url{https://en.wikipedia.org/wiki/Hypervisor}
        \end{itemize}
    \item management tool: libvirt
        \begin{itemize}
            \item libvirt wiki \url{https://en.wikipedia.org/wiki/Libvirt}
            \item Domain XML format \url{http://libvirt.org/formatdomain.html}
            \item Driver capabilities XML format \url{http://libvirt.org/formatcaps.html}
        \end{itemize}
    \item QEMU (hardware emulation, full virtualization)
        \begin{itemize}
            \item QEMU wiki \url{https://en.wikipedia.org/wiki/QEMU}
            \item QEMU wikibook \url{https://en.wikibooks.org/wiki/QEMU}
            \item How to use qemu \url{https://fedoraproject.org/wiki/How_to_use_qemu#Qemu_commands_since_F.3F.2B}
        \end{itemize}
    \item KVM (hardware-assisted virtualization, paravirtualization)
        \begin{itemize}
            \item Kernel-based Virtual Machine wiki \url{https://en.wikipedia.org/wiki/Kernel-based_Virtual_Machine}
            \item Difference between KVM and QEMU \url{http://serverfault.com/questions/208693/difference-between-kvm-and-qemu}
            \item windows virtio drivers \url{https://fedoraproject.org/wiki/Windows_Virtio_Drivers#Direct_download}
            \item QEMU/Windows guest \url{https://wiki.gentoo.org/wiki/QEMU/Windows_guest}
            \item Example using SPICE and QXL for improved Graphics experience in the guest \url{http://www.linux-kvm.org/page/SPICE}
        \end{itemize}
    \item chroot
    \item open container, runC, docker (os-level virtualization)
        \begin{itemize}
            \item Open Container Specifications \url{https://github.com/opencontainers/specs}
            \item OCI FAQs \url{https://www.opencontainers.org/faq}
            \item runC homepage Getting Started \url{https://runc.io/}
            \item runC readme \url{https://github.com/opencontainers/runc}
            \item docker wiki \url{https://en.wikipedia.org/wiki/Docker_(software)}
            \item docker documentation: Get Started with Docker Engine for Linux https://docs.docker.com/linux/
            \item docker documentation: Understand the architecture \url{https://docs.docker.com/engine/understanding-docker/}
            \item docker documentation: Quickstart Docker Engine \url{https://docs.docker.com/engine/quickstart/}
        \end{itemize}
\end{itemize}
%
\subsubsection{File Systems}
%
\begin{itemize}
    \item Union mount, overlayfs
        \begin{itemize}
            \item Union mount https://en.wikipedia.org/wiki/Union_mount
            \item OverlayFS https://en.wikipedia.org/wiki/OverlayFS
            \item kernel documentation https://www.kernel.org/doc/Documentation/filesystems/overlayfs.txt
            \item Arch linux overlayfs wiki https://wiki.archlinux.org/index.php/Overlay_filesystem
        \end{itemize}
\end{itemize}
\subsubsection{Regular Expression}
\begin{enumerate}
    \item Mastering Regular Expression (在读)
\end{enumerate}
%
\subsubsection{Misc}
%
\begin{itemize}
    \item watchdog timer
        \begin{itemize}
            \item watchdog kernel documentation https://www.kernel.org/doc/Documentation/watchdog/watchdog-api.txt
        \end{itemize}
\end{itemize}
%
\subsubsection{History}
%
\begin{itemize}
    \item fedora wiki \url{https://en.wikipedia.org/wiki/Fedora_(operating_system)}
    \item RHEL wiki \url{https://en.wikipedia.org/wiki/Red_Hat_Enterprise_Linux}
    \item the relationship between Fedora and RHEL
        \begin{itemize}
            \item What is the relationship between Fedora and Red Hat Enterprise Linux? \url{https://www.redhat.com/en/technologies/linux-platforms/articles/relationship-between-fedora-and-rhel}
            \item fedora wiki: Red Hat Enterprise Linux {https://fedoraproject.org/wiki/Red_Hat_Enterprise_Linux}
            \item Red Hat Enterprise Linux derivatives \url{https://en.wikipedia.org/wiki/Red_Hat_Enterprise_Linux_derivatives}
        \end{itemize}
    \item Bell Labs \url{https://en.wikipedia.org/wiki/Bell_Labs}
    \item Computer Systems Research Group wiki \url{https://en.wikipedia.org/wiki/Computer_Systems_Research_Group}
    \item Andrew Tanenbaum \url{https://en.wikipedia.org/wiki/Andrew_S._Tanenbaum}
    \item Bill Joy \url{https://en.wikipedia.org/wiki/Bill_Joy}
    \item Novell \url{https://en.wikipedia.org/wiki/Novell}
    \item Unix System Laboratories \url{https://en.wikipedia.org/wiki/Unix_System_Laboratories}
\end{itemize}
\subsection{HTML\&\ CSS}
\begin{enumerate}
    \item \href{https://developer.mozilla.org/en-US/Learn}{Mozilla Developer Network} (在读)
    \item \href{https://developer.mozilla.org/en-US/Learn/Getting_started_with_the_web}{Mozilla Developer Network: Getting Started with Web} (在读)
    \item HTML and XHTML: The Definitive Guide (已读, 有用的东西很多, 废话也很多)
    \item CSS: The Definitive Guide (在读)
\end{enumerate}
%
\subsection{XML}
%
\subsection{JavaScript}
\begin{enumerate}
    \item  \href{http://bonsaiden.github.io/JavaScript-Garden/}{Javascript Garden} (在读)
    \item \href{https://developer.mozilla.org/en-US/Learn}{Mozilla Developer Network} (在读)
    \item JavaScript: The Good Parts (在读)
    \item JavaScript: The Definitive Guide (在读)
\end{enumerate}
\subsection{JSON}
\begin{enumerate}
    \item JSON format intro by official site \url{http://www.json.org/}
\end{enumerate}
\subsection{VCS}
\begin{enumerate}
    \item Pro Git, 2nd edition. (已读)
    \item Version Control with Subversion (在读)
    \item gitlab documentation
        \begin{itemize}
            \item continuous integration (CI)
                \begin{itemize}
                    \item Get started with GitLab CI
                \end{itemize}
        \end{itemize}
\end{enumerate}
\subsection{Computer System}
\begin{enumerate}
    \item Computer system: A programmer's perspective. (在读)
    \item GUID Partition Table wiki \url{https://en.wikipedia.org/wiki/GUID_Partition_Table}
    \item RAID wiki \url{https://en.wikipedia.org/wiki/RAID}
\end{enumerate}
\subsection{Network}
\begin{enumerate}
    \item Sams Teach Yourself TCP/IP in 24 Hours (在读)
    \item TCP/IP Illustrated 2nd (在读)
    \item application layer
        \begin{itemize}
            \item HTTP
                \begin{enumerate}
                    \item HTTP: The Definitive Guide (在读, 很多有价值的参考资料在章末)
                \end{enumerate}
            \item email
                \begin{enumerate}
                    \item RFC2822: Internet Message Format
                \end{enumerate}
            \item file sharing and printing
                \begin{enumerate}
                    \item Network File System Protocol (wikipedia)
                    \item Internet Printing Protocol (Wikipedia)
                    \item CUPS---Common Unix Printing System (Wikipedia)
                    \item network printing from ubuntu \url{https://help.ubuntu.com/community/NetworkPrintingWithUbuntu}
                    \item samba
                        \begin{enumerate}
                            \item samba (Wikipedia)
                            \item samba (ubuntu community wiki)
                            \item samba file server (ubuntu community wiki)
                            \item samba print server (ubuntu community wiki)
                            \item SambaServerGuide (ubuntu community wiki)
                            \item mountWindowsSharesPermanently (ubuntu community wiki)
                        \end{enumerate}
                    \item SMB/CIFS protocol (Wikipedia)
                \end{enumerate}
            \item remote management
                \begin{itemize}
                    \item SNMP
                        \begin{itemize}
                            \item pysnmp doc http://pysnmp.sourceforge.net/docs/snmp-history.html
                            \item snmp wiki https://en.wikipedia.org/wiki/Simple_Network_Management_Protocol
                            \item management information base https://en.wikipedia.org/wiki/Management_information_base
                        \end{itemize}
                \end{itemize}
        \end{itemize}
    \item network layer
        \begin{itemize}
            \item Internet Protocol (IP)
                \begin{itemize}
                    \item Routing selection: specificity vs metric
                        \url{http://serverfault.com/questions/648276/routing-selection-specificity-vs-metric}
                    \item ip address scope parameter
                        \url{http://serverfault.com/questions/63014/ip-address-scope-parameter}
                \end{itemize}
        \end{itemize}
    \item link layer
        \begin{itemize}
            \item IEEE 802.11ac standard \url{https://en.wikipedia.org/wiki/IEEE_802.11ac}
        \end{itemize}
\end{enumerate}
\subsection{Database}
\begin{enumerate}
    \item PostgreSQL Documentation (在读)
    \item Getting Started with MongoDB: Mongo Shell
    \item Getting Started with MongoDB: Python Driver
\end{enumerate}
\subsection{Character Set and Encodings}
\begin{enumerate}
    \item The difference between UTF-8 and Unicode \url{http://www.rrn.dk/the-difference-between-utf-8-and-unicode/}
    \item Unicode HowTo \url{https://docs.python.org/2/howto/unicode.html#python-2-x-s-unicode-support}
\end{enumerate}
%
\subsection{reStructuredText}
\begin{itemize}
    \item a reStructuredText Primer \url{http://docutils.sourceforge.net/docs/user/rst/quickstart.html}
    \item Quick reStructuredText \url{http://docutils.sourceforge.net/docs/user/rst/quickref.html}
    \item An Introduction to reStructuredText \url{http://docutils.sourceforge.net/docs/ref/rst/introduction.html}
\end{itemize}
%
\subsection{Markdown}
\begin{enumerate}
    \item markdown documentation (无笔记)
    \item markdown extra documentation (在读)
\end{enumerate}
\subsection{YAML}
\begin{enumerate}
    \item \href{https://en.wikipedia.org/wiki/YAML}{YAML wiki}
\end{enumerate}
\subsection{Anti-virus, Security, Crypotography}
\begin{enumerate}
    \item yara, documentation \url{http://yara.readthedocs.org/en/v3.4.0/writingrules.html}
    \item Disk encryption introduction \url{https://wiki.archlinux.org/index.php/Disk_encryption}
\end{enumerate}
\section{English Learning}
\subsection{文法}
\begin{itemize}
    \item 文法俱乐部, by 旋元佑. (待复习整理)
    \item 赖世雄经典语法 (在读)
    \item 賴氏英文文法 (在读)
\end{itemize}
\subsection{词汇}
\begin{itemize}
    \item 拓词上的几本词库, 考研词库, 六级词库, 等等.
    \item 欧陆词典, 配合许多词典文件.
    \item 字源大挪移, by 旋元佑. (在读)
    \item Google search, 查询字源.
\end{itemize}
\subsection{修辞}
\section{Not so important}
\subsection{Science Fiction}
\begin{enumerate}
    \item Ender's Game.
\end{enumerate}
\subsection{Fantasy}
\begin{enumerate}
    \item Under the Dome
\end{enumerate}
\subsection{History}

\subsection{Miscellaneous}
\begin{itemize}
    \item 北京---一座失去建筑哲学的城市, by 王博.
    \item Alice's Adventures in Wonderland, by Charles Lutwidge Dodgson.
    \item Steve Jobs, by Walter Isaacson.
\end{itemize}
\end{document}
