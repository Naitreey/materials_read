% arara: xelatex: {action: nonstopmode,synctex: yes}
% arara: biber
% arara: xelatex: {action: nonstopmode,synctex: yes}
% arara: xelatex: {action: nonstopmode,synctex: yes}
\RequirePackage[l2tabu, orthodox]{nag}
\documentclass{article}

%文献宏包
%\usepackage[style=numeric,backend=biber]{biblatex}

\usepackage[a4paper]{geometry}

%字体宏包
%\usepackage[Ligatures=TeX]{fontspec}
%中文宏包
%\usepackage[UTF8,winfonts,fancyhdr,hyperref,fntef]{ctex}
\usepackage{xeCJK}

\usepackage[pdfborder={0 0 0},colorlinks=false]{hyperref}
\usepackage{graphicx,amsmath,amssymb,bm,mathtools,subcaption,caption,cleveref}
\usepackage{IEEEtrantools}
\usepackage{dtklogos}

%数学定义
\renewcommand{\emptyset}{\varnothing}

%single & double quotes
\newcommand{\sq}[1]{`#1'}
\newcommand{\dq}[1]{``#1''}

%文献源 \addbibresource{•}

\newcommand{\book}[1]{\textit{#1}}

%设置字体
%\setmainfont[Ligatures=TeX]{Helvetica}
%\setsansfont{Helvetica}%Arial
%\setmonofont{}
\setCJKmainfont[BoldFont={SimHei},ItalicFont={楷体}]{SimSun}%Hiragino Sans GB W3/W6
%\setCJKsansfont{Hiragino Sans GB W3}
\setCJKmonofont{幼圆}

%代码宏包
%\usepackage{listings}
%\lstset{language=,frame=single,texcl=true,mathescape=true}%[LaTeX]TeX or mathematica or etc?

\title{Materials That I Have Read All These Years}
\author{Naitree Zhu}
\date{Last modified \today}

\begin{document}
\maketitle
%目录 \tableofcontents

\section{Physics}
\subsection{经典力学}
\begin{itemize}
    \item \book{普通物理学教程: 力学} (第二版), 漆安慎, 杜婵英. (在读)
    \item \book{新概念物理教程: 力学} (第二版), 赵凯华, 罗蔚茵. (在读)
    \item \book{An Introduction to Mechanics}, 2nd, by Kleppner and Kolenkow. (在读)
    \item \href{http://physics.stackexchange.com/}{Physics.SE}
        \begin{itemize}
            \item \href{http://physics.stackexchange.com/questions/1984/why-does-holding-something-up-cost-energy-while-no-work-is-being-done}{Why does holding something up cost energy while no work is being done?}
            \item \href{http://physics.stackexchange.com/questions/132978/who-does-work-while-walking}{Who does work while walking?}
        \end{itemize}
\end{itemize}

\subsection{热物理学}
\begin{itemize}
    \item \book{新概念物理教程: 热学} (第二版), 赵凯华, 罗蔚茵. (在读)
    \item \book{热物理学基础}, 包柯达. (在读)
    \item \book{热学}, 秦允豪. (在读)
    \item \book{Heat and Thermodynamics}, 7th edition, by Zemansky. (在读)
    \item \book{Thermodynamics and An Introduction To Thermostatistics}, 2nd edition, by Callen. (在读)
\end{itemize}

\subsection{电磁理论}
\begin{enumerate}
    \item 普通物理学教程: 电磁学, 第二版, by 梁灿彬. (在读)
\end{enumerate}
\subsection{量子理论}
\begin{itemize}
    \item \href{http://physics.stackexchange.com/}{Physics.SE}
        \begin{itemize}
            \item \href{http://physics.stackexchange.com/questions/133270/can-someone-please-qualitatively-explain-unitary-group-from-a-physics-perspectiv}{Can someone please qualitatively explain unitary group from a physics perspective?}
        \end{itemize}
\end{itemize}
\section{Mathematics}
\subsection{General Materials}
\begin{itemize}
    \item \href{http://mathworld.wolfram.com/}{Wolfram MathWorld}
    \item Wikipedia
    \item Mathematics Stackexchange
    \item Mathoverflow
\end{itemize}
\subsection{分析学}
\begin{itemize}
    \item \book{高等数学}, 同济第六版.
    \item \book{数学分析新讲}, 第一册, 张筑生. (在读)
\end{itemize}
\section{Computer Science}
%
\subsection{Learned Languages}
\begin{itemize}
    %programming languages
    \item C
    \item Python
    \item Java
    \item Javascript
    \item Mathematica
    \item bash/shell script
    \item AWK
    \item TeX
    \item M4
    %markup languages
    \item TeX
    \item HTML/XML
    \item CSS
    \item reStructuredText
    \item Markdown
    %database languages
    \item SQL
    \item MongoDB
    \item Elasticsearch
    %data format
    \item XML
    \item YAML
    \item JSON
\end{itemize}
%
\subsection{TeX}
\begin{itemize}
    \item \LaTeX{} Wikibook.
    \item \LaTeX{} Stackexchange
    \item The Not So Short Introduction to \LaTeXe{}.
    \item arara user manual. (无笔记)
    \item listings package documentation. (未读完)
    \item geometry package documentation.
    \item pdfpages package documentation. (未读完)
    \item amsmath package documentation.
    \item siunitx package documentation.
    \item lnotes, by 包太雷.
    \item lnotes2, by 包太雷.
    \item booktabs package documentation.
    \item \href{http://tex.stackexchange.com/}{TeX.SE}
        \begin{itemize}
            \item \href{http://tex.stackexchange.com/questions/198613/spacing-around-align-environments}{Spacing around align environments}
            \item the TeXbook. (在读)
            \item pgfmanual. (在读)
            \item enumitem package documentation. (在读)
            \item cool package doc \& demo. (在读)
        \end{itemize}
\end{itemize}
\subsection{Vim}
\begin{itemize}
    \item Vim tutor.
    \item Vim documentations
        \begin{itemize}
            \item usr_40.txt (在读)
            \item if_cscop.txt (cscope)
        \end{itemize}
    \item Vim-LaTeX reference. (剩下后面需要理解 vimscript 的部分)
    \item Vim-LaTeX reference card.
    \item \book{Practical Vim}, by Drew Neil. (在读)
    \item \href{http://vim.wikia.com/wiki/Vim_Tips_Wiki}{Vim Tips wiki}
    \item Vim questions on StackOverflow
    \item vim features and commands:
        \begin{itemize}
            \item movement
                h, j, k, l, w, W, e, E, \^, \$, gg, G, {n}G, /{char}, ?{char}, n, N, *,
                <C-o>, <C-i>, f, F, t, T, ;, \verb|,|, zz,
            \item editing
                \begin{itemize}
                    \item cmdline mode
                        :w, :wa, :wq, :wqa, <C-p>, <C-n>, \%, :h, :sort
                    \item normal mode
                        a, A, I, i, x, d, dd, D, gU, gu, u, <C-r>, r, R, c, C, s, S,
                        v, V, <C-v>, o, O, y, yy, Y, >, >>, <, <<, ., <C-a>, <C-x>
                    \item insert mode
                        <C-k>, <C-v>, <ESC>,
                \end{itemize}
            \item info
                <C-g>, ga,
            \item session
                <C-z>
            \item tab
                \begin{itemize}
                    \item cmdline mode
                        :tabnew, :tabclose
                        :tabfirst, :tablast, :tabnext, :tabprevious
                    \item normal mode
                        gt, gT
                \end{itemize}
            \item working directory
                :lcd, :cd
        \end{itemize}
\end{itemize}
\subsection{Mathematica}
\begin{itemize}
    \item \book{Mathematica 与大学物理计算}. (完全不合理地使用了 MMA)
    \item \book{The Mathematica Book}, by Stephen Wolfram. (在读)
    \item Mathematica virtual book---help doc. (在读)
    \item Mathematica Cookbook, by Sal Mangano (在读)
    \item Power programming with Mathematica: The kernel. (在读)
    \item Video tutorials: \href{http://www.wolfram.com/broadcast/}{Wolfram Screencast \&{} Video Gallery} (在读)
    \item \href{http://demonstrations.wolfram.com/}{Wolfram Demonstrations Project} (在读)
    \item \href{http://stoney.sb.org/wordpress/2008/12/adding-a-keyboard-shortcut-to-mathematica-v7/}{修改默认的 brackets 插入快捷键使得可以 wrap text selection}
    \item Mathematica Stackexchange:
        \begin{itemize}
            \item \href{http://mathematica.stackexchange.com/questions/18393/what-are-the-most-common-pitfalls-awaiting-new-users}{What are the most common pitfalls awaiting new users?}
            \item \href{http://mathematica.stackexchange.com/questions/24988/can-one-identify-the-design-patterns-of-mathematica}{Can one identify the design patterns of Mathematica?}
            \item \href{http://mathematica.stackexchange.com/questions/8310/using-the-mathematica-front-end-efficiently-for-editing-notebooks}{Using the Mathematica front-end efficiently for editing notebooks}
            \item \href{http://mathematica.stackexchange.com/questions/853/is-there-a-way-to-separate-variables-between-multiple-notebooks?lq=1}{Is there a way to separate variables between multiple notebooks?}
            \item \href{http://mathematica.stackexchange.com/questions/57656/how-draw-a-line-dynamically-using-manipulate}{How draw a line dynamically using manipulate?}
        \end{itemize}
    \item Wolfram blog:
        \begin{itemize}
            \item \href{http://blog.wolfram.com/2011/12/07/10-tips-for-writing-fast-mathematica-code/}{10 Tips for Writing Fast Mathematica Code}
        \end{itemize}
\end{itemize}
\subsection{Java}
\begin{enumerate}
    \item Java: A Beginner's Guide (在读)
    \item Java Language Specification (在读)
    \item \href{http://docs.oracle.com/javase/8/docs/api/index.html?overview-summary.html}{Java Platform Standard Edition 8 API doc} (在读)
    \item \href{http://tomcat.apache.org/tomcat-8.0-doc/servletapi/}{Apache Tomcat Servlet API doc} (在读)
    \item \href{http://logging.apache.org/log4j/2.x/log4j-api/apidocs/index.html}{Apache Log4j API doc} (在读)
    \item How to Write Doc Comments for the Javadoc Tool (在读)
    \item Spring by Example (在读)
    \item Spring Framework Reference Documentation (在读)
    \item Tomcat tutorial (已读)
\end{enumerate}
%
\subsection{Go}
%
\begin{itemize}
\end{itemize}
%
\subsection{Operating System}
%
\begin{itemize}
    \item trap instruction wiki
          \url{https://en.wikipedia.org/wiki/Trap_(computing)}
    \item data segment wiki
          \url{https://en.wikipedia.org/wiki/Data_segment}
    \item Anatomy of a Program in Memory
          \url{http://duartes.org/gustavo/blog/post/anatomy-of-a-program-in-memory/}
\end{itemize}
\subsection{HTML\&\ CSS}
\begin{enumerate}
    \item \href{https://developer.mozilla.org/en-US/Learn}{Mozilla Developer Network} (在读)
    \item \href{https://developer.mozilla.org/en-US/Learn/Getting_started_with_the_web}{Mozilla Developer Network: Getting Started with Web} (在读)
    \item HTML and XHTML: The Definitive Guide (已读, 有用的东西很多, 废话也很多)
    \item CSS: The Definitive Guide (在读)
\end{enumerate}
%
\subsection{XML}
%
\subsection{JSON}
\begin{enumerate}
    \item JSON format intro by official site \url{http://www.json.org/}
\end{enumerate}
\subsection{VCS}
\begin{enumerate}
    \item Version Control with Subversion (在读)
\end{enumerate}
%
\subsubsection{Git}
%
\begin{itemize}
    \item Pro Git, 2nd edition. (已读)
    \item Git Tip of the Week: Git Notes
        \url{http://alblue.bandlem.com/2011/11/git-tip-of-week-git-notes.html}
    \item gitlab documentation
        \begin{itemize}
            \item continuous integration (CI)
                \begin{itemize}
                    \item Get started with GitLab CI
                \end{itemize}
        \end{itemize}
\end{itemize}
%
\subsection{Character Set and Encodings}
\begin{enumerate}
    \item The difference between UTF-8 and Unicode \url{http://www.rrn.dk/the-difference-between-utf-8-and-unicode/}
    \item Unicode HowTo \url{https://docs.python.org/2/howto/unicode.html#python-2-x-s-unicode-support}
\end{enumerate}
%
\subsection{reStructuredText}
\begin{itemize}
    \item a reStructuredText Primer \url{http://docutils.sourceforge.net/docs/user/rst/quickstart.html}
    \item Quick reStructuredText \url{http://docutils.sourceforge.net/docs/user/rst/quickref.html}
    \item An Introduction to reStructuredText \url{http://docutils.sourceforge.net/docs/ref/rst/introduction.html}
\end{itemize}
%
\subsection{Markdown}
\begin{enumerate}
    \item markdown documentation (无笔记)
    \item markdown extra documentation (在读)
\end{enumerate}
\subsection{YAML}
\begin{enumerate}
    \item \href{https://en.wikipedia.org/wiki/YAML}{YAML wiki}
\end{enumerate}
\section{English Learning}
\subsection{文法}
\begin{itemize}
    \item 文法俱乐部, by 旋元佑. (待复习整理)
    \item 赖世雄经典语法 (在读)
    \item 賴氏英文文法 (在读)
\end{itemize}
\subsection{词汇}
\begin{itemize}
    \item 拓词上的几本词库, 考研词库, 六级词库, 等等.
    \item 欧陆词典, 配合许多词典文件.
    \item 字源大挪移, by 旋元佑. (在读)
    \item Google search, 查询字源.
\end{itemize}
\subsection{修辞}
\section{Not so important}
\subsection{Science Fiction}
\begin{enumerate}
    \item Ender's Game.
\end{enumerate}
\subsection{Fantasy}
\begin{enumerate}
    \item Under the Dome
\end{enumerate}
\subsection{History}

\subsection{Miscellaneous}
\begin{itemize}
    \item 北京---一座失去建筑哲学的城市, by 王博.
    \item Alice's Adventures in Wonderland, by Charles Lutwidge Dodgson.
    \item Steve Jobs, by Walter Isaacson.
\end{itemize}
\end{document}
